% \iffalse 
%
% Copyright (C) 2020 by Andres Merino <mat.andresmerino@gmail.com>
% 
% Para un mejor uso y entendimiento del actual paquete, consultar la documentación.
% -----------------------------------------------------------
%
% \fi
%
%  \section{Implementación}
%  \subsection{Identificación}
%  Dado que esta clase utiliza el comando \cmd{\RequirePackage}, no funciona con 
%  versiones antiguas de \LaTeXe.
%    \begin{macrocode}
\NeedsTeXFormat{LaTeX2e}[2009/09/24]
%    \end{macrocode}
%  El paquete se identifica con su fecha de lanzamiento y su número de versión.
%  \begin{macrocode}
\ProvidesPackage{aleph-comandos}[2020/08/14 v1.1]
%    \end{macrocode}
%  \subsection{Paquetes}
% \iffalse
%%%%%%%%%%%%%%%%%%%%%%%%%%%%%%%%%%%%
%% --- Paquetes
%%%%%%%%%%%%%%%%%%%%%%%%%%%%%%%%%%%%
% \fi
%%  Son necesarios los siguientes paquetes para utilizar los comandos.
%    \begin{macrocode}
\RequirePackage{ifthen}
\RequirePackage{calc}
\RequirePackage{etex}
\RequirePackage{amsmath,amssymb}
\RequirePackage{xcolor}
%    \end{macrocode}
%  \subsection{Comandos de función}
% \iffalse
%%%%%%%%%%%%%%%%%%%%%%%%%%%%%%%%%%%%
%% --- Comandos de función
%%%%%%%%%%%%%%%%%%%%%%%%%%%%%%%%%%%%
% \fi
%%  Función completa
%    \begin{macrocode}
\newcommand{\funcion}[5]{%
    {\setlength{\arraycolsep}{2pt}
    \begin{array}{r@{\,}ccl}
        #1\colon & #2 & \longrightarrow & #3\\
                & #4 & \longmapsto & \displaystyle#5
    \end{array}
    }
}
%    \end{macrocode}
%%  Función dom-img
%    \begin{macrocode}
\newcommand{\func}[3]{ #1\colon #2 \rightarrow  #3}
%    \end{macrocode}
%  \subsection{Conjuntos}
% \iffalse
%%%%%%%%%%%%%%%%%%%%%%%%%%%%%%%%%%%%
%% --- Conjuntos
%%%%%%%%%%%%%%%%%%%%%%%%%%%%%%%%%%%%
% \fi
%%  Números naturales
%    \begin{macrocode}
\newcommand{\N}{\mathbb{N}}
\newcommand{\Nbb}{\mathbb{N}}
%    \end{macrocode}
%%  Números enteros
%    \begin{macrocode}
\newcommand{\Z}{\mathbb{Z}}
\newcommand{\Zbb}{\mathbb{Z}}
%    \end{macrocode}
%%  Números racionales
%    \begin{macrocode}
\newcommand{\Q}{\mathbb{Q}}
\newcommand{\Qbb}{\mathbb{Q}}
%    \end{macrocode}
%%  Números reales
%    \begin{macrocode}
\newcommand{\R}{\mathbb{R}}
\newcommand{\Rbb}{\mathbb{R}}
\newcommand{\reales}{\mathbb{R}}
%    \end{macrocode}
%%  Números complejos
%    \begin{macrocode}
\@ifundefined{C}
    {\newcommand{\C}{\mathbb{C}}}
    {\renewcommand{\C}{\mathbb{C}}}
\newcommand{\Cbb}{\mathbb{C}}
%    \end{macrocode}
%%  Campos
%    \begin{macrocode}
\newcommand{\K}{\mathbb{K}}
\newcommand{\Kbb}{\mathbb{K}}
%    \end{macrocode}
%%  Primos
%    \begin{macrocode}
\newcommand{\Pbb}{\mathbb{P}}
%    \end{macrocode}
%%  Polinomios
%    \begin{macrocode}
\newcommand{\Pol}{\mathcal{P}}
%    \end{macrocode}
%%  Matrices
%    \begin{macrocode}
\newcommand{\M}{\mathcal{M}}
%    \end{macrocode}
%%  Matrices 2
%    \begin{macrocode}
\newcommand{\Mat}[3][\R]{#1^{#2\times #3}}
%    \end{macrocode}
%%  Números irracionales
%    \begin{macrocode}
\newcommand{\Ibb}{\mathbb{I}}
%    \end{macrocode}
%  \subsection{Operadores}
% \iffalse
%%%%%%%%%%%%%%%%%%%%%%%%%%%%%%%%%%%%
%% --- Operadores
%%%%%%%%%%%%%%%%%%%%%%%%%%%%%%%%%%%%
% \fi
%%  Dominio
%    \begin{macrocode}
\DeclareMathOperator{\dom}{dom}
\DeclareMathOperator{\Dom}{Dom}
%    \end{macrocode}
%%  Recorrido
%    \begin{macrocode}
\DeclareMathOperator{\rec}{rec}
\DeclareMathOperator{\Rec}{Rec}
%    \end{macrocode}
%%  Imagen
%    \begin{macrocode}
\DeclareMathOperator{\img}{img}
\DeclareMathOperator{\Img}{Img}
%    \end{macrocode}
%%  Rango de una matriz
%    \begin{macrocode}
\DeclareMathOperator{\rg}{rg}
\DeclareMathOperator{\rang}{rang}
%    \end{macrocode}
%%  Matriz adjunta
%    \begin{macrocode}
\DeclareMathOperator{\adj}{adj}
%    \end{macrocode}
%%  Matriz de cofactores 
%    \begin{macrocode}
\DeclareMathOperator{\cof}{cof}
%    \end{macrocode}
%%  Espacio generado
%    \begin{macrocode}
\DeclareMathOperator{\gen}{gen}
%    \end{macrocode}
%%  Proyección
%    \begin{macrocode}
\DeclareMathOperator{\proy}{proy}
%    \end{macrocode}
%%  Componente normal
%    \begin{macrocode}
\DeclareMathOperator{\norm}{norm}
%    \end{macrocode}
%%  Interior de un conjunto
%    \begin{macrocode}
\DeclareMathOperator{\inte}{int}
%    \end{macrocode}
%%  Trigonométricas
%    \begin{macrocode}
\renewcommand{\sin}{\sen}
%    \end{macrocode}
%%  Trigonométricas inversa
%    \begin{macrocode}
\let\arctan\relax
\DeclareMathOperator{\arctan}{arc\,tan}
\DeclareMathOperator{\arccsc}{arc\,csc}
\DeclareMathOperator{\arccot}{arc\,cot}
\DeclareMathOperator{\arcsec}{arc\,sec}
\DeclareMathOperator{\arcsen}{arc\,sen}
\let\arccos\relax
\DeclareMathOperator{\arccos}{arc\,cos}
\let\arcsin\relax
\DeclareMathOperator{\arcsin}{arc\,sen}
%    \end{macrocode}
%%  Espacio generado
%    \begin{macrocode}
\DeclareMathOperator{\spn}{span}
%    \end{macrocode}
%%  Parte real y parte imaginaria
%    \begin{macrocode}
\DeclareMathOperator{\im}{Im}
\DeclareMathOperator{\re}{Re}
%    \end{macrocode}
%%  Gráfico de una función
%    \begin{macrocode}
\DeclareMathOperator{\graf}{graf}
%    \end{macrocode}
%%  Operador signo
%    \begin{macrocode}
\DeclareMathOperator{\sgn}{sgn}
%    \end{macrocode}
%%  Conjunto de valores admisible
%    \begin{macrocode}
\DeclareMathOperator{\CVA}{CVA}
%    \end{macrocode}
%%  Conjunto solución
%    \begin{macrocode}
\DeclareMathOperator{\Sol}{Sol}
\DeclareMathOperator{\sol}{Sol}
%    \end{macrocode}
%%  Operador cis (cos + i sen)
%    \begin{macrocode}
\DeclareMathOperator{\Cis}{Cis}
\DeclareMathOperator{\cis}{Cis}
%    \end{macrocode}
%%  Diámetro
%    \begin{macrocode}
\DeclareMathOperator{\diam}{diam}
%    \end{macrocode}
%%  Varianza
%    \begin{macrocode}
\DeclareMathOperator{\Var}{Var}
%    \end{macrocode}
%%  Traza
%    \begin{macrocode}
\DeclareMathOperator{\Tr}{tr}
\DeclareMathOperator{\tr}{tr}
%    \end{macrocode}
%%  Máximo común divisor
%    \begin{macrocode}
\DeclareMathOperator{\mcd}{mcd}
%    \end{macrocode}
%%  Mínimo común múltiplo
%    \begin{macrocode}
\DeclareMathOperator{\mcm}{mcm}
%    \end{macrocode}
%%  Divergencia
%    \begin{macrocode}
\DeclareMathOperator{\dive}{div}
%    \end{macrocode}
%%  Rotacional
%    \begin{macrocode}
\DeclareMathOperator{\rot}{rot}
%    \end{macrocode}
%%  Partes de un conjunto
%    \begin{macrocode}
\DeclareMathOperator{\partes}{\mathcal{P}}
%    \end{macrocode}
%  \subsection{Operadores como comandos}
% \iffalse
%%%%%%%%%%%%%%%%%%%%%%%%%%%%%%%%%%%%
%% --- Operadores como comandos
%%%%%%%%%%%%%%%%%%%%%%%%%%%%%%%%%%%%
% \fi
%%  Clausura de un conjunto
%    \begin{macrocode}
\newcommand{\cl}[1]{\overline{#1}}
%    \end{macrocode}
%%  Norma
%    \begin{macrocode}
\newcommand{\norma}[1]{%
    \left\|\ifthenelse{\equal{#1}{}}
                  {\cdot}{#1}
    \right\|}
%    \end{macrocode}
%%  Producto interno
%    \begin{macrocode}
\newcommand{\prodinner}[2]{%
    \left\langle\ifthenelse{\equal{#1}{}\and\equal{#2}{}}
               {\cdot,\cdot}
               {#1,\, #2}
    \right\rangle}
%    \end{macrocode}
%%  Conjugado
%    \begin{macrocode}
\newcommand{\conjugate}[1]{\overline{#1}}
%    \end{macrocode}
%%  Derivada parcial
%    \begin{macrocode}
\newcommand{\parcial}[2]{\dfrac{\partial #1 }{\partial #2}}
%    \end{macrocode}
%%  Derivada total
%    \begin{macrocode}
\newcommand{\derivada}[2]{\dfrac{d #1 }{d #2}}
%    \end{macrocode}
%  \subsection{Abreviaciones}
% \iffalse
%%%%%%%%%%%%%%%%%%%%%%%%%%%%%%%%%%%%
%% --- Abreviaciones
%%%%%%%%%%%%%%%%%%%%%%%%%%%%%%%%%%%%
% \fi
%%  Diferencia de conjuntos pequeña
%    \begin{macrocode}
\renewcommand{\setminus}{\smallsetminus}
%    \end{macrocode}
%%  Contenecia de conjuntos con igual
%    \begin{macrocode}
\newcommand{\sset}{\subseteq}
%    \end{macrocode}
%%  Conjunto vacío
%    \begin{macrocode}
\renewcommand{\emptyset}{\varnothing}
%    \end{macrocode}
%%  Épsilon
%    \begin{macrocode}
\newcommand{\vepsilon}{\varepsilon}
%    \end{macrocode}
%%  Texto ``y'' con espacio
%    \begin{macrocode}
\newcommand{\texty}{\qquad\text{y}\qquad}
\newcommand{\yds}{\quad\text{y}\quad}
%    \end{macrocode}
%%  Texto ``o'' con espacio
%    \begin{macrocode}
\newcommand{\texto}{\qquad\text{o}\qquad}
\newcommand{\ods}{\quad\text{o}\quad}
%    \end{macrocode}
%%  Texto ``si y solo si'' con espacio
%    \begin{macrocode}
\newcommand{\siysolosi}{\quad\text{si y solo si}\quad}
\newcommand{\ssi}{\quad\text{si y solo si}\quad}
%    \end{macrocode}
%%  Grados
%    \begin{macrocode}
\newcommand{\degre}{\ensuremath{^\circ}}
\newcommand{\grad}{\ensuremath{^\circ}}
%    \end{macrocode}
%  \subsection{Comandos desplegados}
% \iffalse
%%%%%%%%%%%%%%%%%%%%%%%%%%%%%%%%%%%%
%% --- Comandos desplegados
%%%%%%%%%%%%%%%%%%%%%%%%%%%%%%%%%%%%
% \fi
%%  Límite en formato desplegado
%    \begin{macrocode}
\newcommand{\dlim}{\displaystyle\lim}
\newcommand{\Lim}{\displaystyle\lim}
%    \end{macrocode}
%%  Sumatoria en formato desplegado
%    \begin{macrocode}
\newcommand{\dsum}{\displaystyle\sum}
\newcommand{\Sum}{\displaystyle\sum}
%    \end{macrocode}
%%  Binomio en formato desplegado
%    \begin{macrocode}
\newcommand{\Binom}{\displaystyle\binom}
%    \end{macrocode}
%%  Integral en formato desplegado
%    \begin{macrocode}
\newcommand{\dint}{\displaystyle\int}
\newcommand{\Int}{\displaystyle\int}
%    \end{macrocode}
%  \subsection{Abreviaciones de operadores lógicos}
% \iffalse
%%%%%%%%%%%%%%%%%%%%%%%%%%%%%%%%%%%%
%% --- Abreviaciones de operadores lógicos
%%%%%%%%%%%%%%%%%%%%%%%%%%%%%%%%%%%%
% \fi
%%  Doble implicación
%    \begin{macrocode}
\newcommand{\Di}{\Longleftrightarrow}
\newcommand{\dimp}{\Leftrightarrow}
\newcommand{\Dimp}{\Longleftrightarrow}
\newcommand{\qDimp}{\quad\Longleftrightarrow\quad}
%    \end{macrocode}
%%  Implicación
%    \begin{macrocode}
\newcommand{\Imp}{\Longrightarrow}
\newcommand{\imp}{\Rightarrow}
\newcommand{\qImp}{\quad\Longrightarrow\quad}
%    \end{macrocode}
%%  Conectores con espacio
%    \begin{macrocode}
\newcommand{\qland}{\quad \land \quad }
\newcommand{\qlor}{\quad \lor \quad }
\newcommand{\orm}{\quad \vee \quad }
\newcommand{\andm}{\quad \wedge \quad }
%    \end{macrocode}
%%  Tautología y contradicción
%    \begin{macrocode}
\newcommand{\V}{\mathbb{V}}
\newcommand{\F}{\mathbb{F}}
%    \end{macrocode}
%  \subsection{Delimitadores}
% \iffalse
%%%%%%%%%%%%%%%%%%%%%%%%%%%%%%%%%%%%
%% --- Delimitadores
%%%%%%%%%%%%%%%%%%%%%%%%%%%%%%%%%%%%
% \fi
%%  Intervalo abierto izquierda
%    \begin{macrocode}
\newcommand{\lop}{\left]}
%    \end{macrocode}
%%  Intervalo cerrado izquierda
%    \begin{macrocode}
\newcommand{\lcl}{\left[}
%    \end{macrocode}
%%  Intervalo abierto derecha
%    \begin{macrocode}
\newcommand{\rop}{\right[}
%    \end{macrocode}
%%  Intervalo cerrado derecha
%    \begin{macrocode}
\newcommand{\rcl}{\right]}
%    \end{macrocode}
%%  Izquierda
%    \begin{macrocode}
\renewcommand{\l}{\left}
%    \end{macrocode}
%%  Derecha
%    \begin{macrocode}
\renewcommand{\r}{\right}
%    \end{macrocode}
%%  Intervalos
%    \begin{macrocode}
\newcommand{\open}[1]{\left]#1\right[}
\newcommand{\openl}[1]{\left]#1\right]}
\newcommand{\openr}[1]{\left[#1\right[}
\newcommand{\close}[1]{\left[#1\right]}
%    \end{macrocode}
%  \subsection{Sucesiones}
% \iffalse
%%%%%%%%%%%%%%%%%%%%%%%%%%%%%%%%%%%%
%% --- Sucesiones
%%%%%%%%%%%%%%%%%%%%%%%%%%%%%%%%%%%%
% \fi
%%  Sucesiones
%    \begin{macrocode}
\newcommand{\suc}[2][n]{\left(#2\right)_{#1\in\mathbb{\N}}}
%    \end{macrocode}
%%  Sucesiones con llaves
%    \begin{macrocode}
\newcommand{\sucl}[2][n]{\left\{#2\right\}_{#1\in\mathbb{\N}}}
%    \end{macrocode}
%  \subsection{Comentarios}
% \iffalse
%%%%%%%%%%%%%%%%%%%%%%%%%%%%%%%%%%%%
%% --- Comentarios
%%%%%%%%%%%%%%%%%%%%%%%%%%%%%%%%%%%%
% \fi
%%  Comentarios
%    \begin{macrocode}
\newcommand{\comentario}[1]{\textcolor{red}{#1}}
%    \end{macrocode}
%  \subsection{Vectores}
% \iffalse
%%%%%%%%%%%%%%%%%%%%%%%%%%%%%%%%%%%%
%% --- Vectores
%%%%%%%%%%%%%%%%%%%%%%%%%%%%%%%%%%%%
% \fi
%%  Vectores canónicos
%    \begin{macrocode}
\newcommand{\veci}{\mathbf{i}}
\newcommand{\vecj}{\mathbf{j}}
\newcommand{\veck}{\mathbf{k}}
%    \end{macrocode}
%  \subsection{Formato}
% \iffalse
%%%%%%%%%%%%%%%%%%%%%%%%%%%%%%%%%%%%
%% --- Formato
%%%%%%%%%%%%%%%%%%%%%%%%%%%%%%%%%%%%
% \fi
%%  Formato
%    \begin{macrocode}
\allowdisplaybreaks
%    \end{macrocode}
% \iffalse
%%  Y ¡se acabó!
% \fi
%    \Finale
\endinput